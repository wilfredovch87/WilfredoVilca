\section{¿Qué es Desarrollo Orientado A Pruebas (TDD)?} 
\textbf{}\\
Esta técnica llamada TDD (Test Driven Development), se puede definir como un proceso de desarrollo de software que se basa en la idea de desarrollar unas pequeñas pruebas, codificarlas y luego refactorizar el código que hemos implementado anteriormente.
Podemos decir que esta técnica e implementación de software está dentro de la metodología XP donde deberíamos de echarle un ojo a todas sus técnicas, tras leer varios artículos en un coincido con Peter Provost con un diseño dirigido o implementado a base de ejemplos hubiese sido mejor pero TDD se centra en 3 objetivos claros:

\begin{flushleft}

\begin{itemize}

	\item Una implementación de las funciones justas que el cliente necesita y no más, solamente las funciones que necesitamos, estoy cansado de duplicar dichas funciones para que hagan lo mismo

	 Mínimos defectos en fase de producción

	\item Producción de software modular y sobre todo reutilizable y preparado para el cambio


\textbf{}\\
Esta técnica se basa en la idea de realizar unas pruebas unitarias para un código que nosotros debemos construir, Nuestro TDD lo que nos dice es que primero los programadores debemos realizar una prueba y a continuación empezar a desarrollar el código que la resuelve.
El método que debemos seguir a para empezar a utilizar TDD es sencillo, Nos sirve para elegir uno de los requisitos a implementar, buscar un primer ejemplo sencillo, crear una prueba, ejecutarla e implementar el código mínimo para superar dicha prueba.
Obviamente la gracia de ejecutar la prueba después de crearla es ver que esta falla y que será necesario hacer algo en el código para que esta pase.
El ciclo de desarrollo de TDD es empezar la prueba, en test realizar un test, revisar el código y pasar el refactor.

\textbf {Crear la prueba o test}
\item Ejecutar los tests: falla (ROJO)
\item Crear código específico para resolver el test
\item Ejecutar de nuevo los tests: pasa (VERDE)
\item Refactorizar el código
\item Ejecutar los tests: pasa (VERDE)

\textbf {Personalmente, añadiría lo siguiente:}
\item Incrementa la productividad.
\item Nos hace descubrir y afrontar más casos de uso en tiempo de diseño.
\item La jornada se hace mucho más amena.
\item Uno se marcha a casa con la reconfortante sensación de que el trabajo está bien hecho.


Ahora bien, como cualquier técnica, no es una varita mágica y no dará el mismo resultado a un experto arquitecto de software que a un programador junior que está empezando. Sin embargo, es útil para ambos y para todo el rango de integrantes del equipo que hay entre uno y otro.
Es una técnica a tener en cuenta en el desarrollo web y sobre todo en el desarrollo de ingeniería software donde debemos tener en cuenta muchos fallos antes de pasar a producción.



\end{itemize} 


\end{flushleft}